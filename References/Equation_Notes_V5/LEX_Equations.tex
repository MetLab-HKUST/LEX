\documentclass[a4paper,11pt]{article}

\usepackage[left=2.5cm,top=2.5cm,right=2.5cm,bottom=2.5cm]{geometry}
\usepackage{graphicx}
\usepackage[font=footnotesize,labelfont=bf]{caption}
\usepackage{enumitem}
\usepackage{xcolor}
\usepackage{algorithm2e}

%\usepackage{amsmath,amsthm} 
%\usepackage{newtxtext}
%\usepackage{newtxmath}
% 
% \usepackage[p,osf]{scholax}
% T1 and textcomp are loaded by package. Change that here, if you want
% load sans and typewriter packages here, if needed
% \usepackage{amsmath,amsthm}% must be loaded before newtxmath
% amssymb should not be loaded
% \usepackage[scaled=1.075,ncf,vvarbb]{newtxmath}% need to scale up math package
% vvarbb selects the STIX version of blackboard bold.

% \usepackage[T1]{fontenc}
% \usepackage{stix}

\usepackage{kpfonts}
\usepackage[T1]{fontenc}

% 
% \usepackage{libertinus}
% \usepackage[T1]{fontenc}
% \renewcommand*\familydefault{\sfdefault} %% Only if the base font of the document is to be sans serif

\setlength{\parskip}{3ex}
\setlength{\parindent}{1.5em}
\renewcommand{\baselinestretch}{1.5}

\usepackage{amsmath,amsthm}
\usepackage[most]{tcolorbox}

\tcbset{colback=yellow!10!white, colframe=red!50!black, 
        highlight math style= {enhanced, %<-- needed for the ’remember’ options
            colframe=red,colback=red!10!white,boxsep=0pt}
        }
        

% Title Page
\title{\textbf{LEX Governing Equations}}
\author{Xiaoming Shi}

\begin{document}
\maketitle

The governing equations we adopted are the acoustic-wave-filtered equations for compressible stratified flow by Durran (2008). A pseudo-density $\rho^*$ is defined to eliminate sound waves. Mass conservation is enforced with respect to this pseudo-density such at
\begin{equation}
 \frac{1}{\rho^*} \frac{D \rho^*}{D t} + \nabla\cdot \mathbf{u} = 0 \nonumber
\end{equation}

In Durran (2008), the pseudo-density is defined as 
\begin{equation}
 \rho^* = \frac{\tilde{\rho}(x,y,z,t)\tilde{\theta}(x,y,z,t)}{\theta} \nonumber 
\end{equation}
where $\tilde{\ }$ denotes a spatially varying reference state. With this definition, the mass (pseudo-density) conservation equation becomes, with some approximation,
\begin{equation}
\tcboxmath{
\frac{\partial \tilde{\rho} \tilde{\theta}}{\partial t} + \nabla\cdot (\tilde{\rho}\tilde{\theta}\mathbf{u}) = \frac{\tilde{\rho} H_m}{c_p \tilde{\pi}}
}
\end{equation}
in which, $H_m$ is the heating rate per unit mass. 

Define perturbations with respect to the reference state such that $\theta' = \theta - \tilde{\theta}$ and $\pi' = \pi - \tilde{\pi}$. Durran (2018) further separated $\tilde{\pi}$ into a large horizontally uniform component $\tilde{\pi}_v(z,t)$ and a remainder $\tilde{\pi}_h(x,y,z,t)$ for computational accuracy and notational convenience. Then the momentum and thermodynamics equations are the following,
\begin{equation}
\tcboxmath{
\frac{D \mathbf{u}_h}{D t}  + f {\mathbf{k}} \times \mathbf{u}_h +  c_p \theta\,\nabla_h(\tilde{\pi}_h + \pi') = 0
}
\end{equation}
\begin{equation}
 \tcboxmath{
 \frac{D w}{D t} + c_p \theta \frac{\partial \pi'}{\partial z} = B%g\frac{\theta'}{\tilde{\theta}}
 }
\end{equation}
\begin{equation}
\tcboxmath{
 \frac{D \theta }{D t} = \frac{H_m}{c_p \tilde{\pi}}
 }
\end{equation}
where $\mathbf{u}_h$ is the horizontal velocity vector, $\nabla_h$ is the horizontal gradient operator, and $f$ is the Coriolis parameter. $B$ is the linearized bouancy,
\begin{equation}
B = g\left[\frac{\theta'}{\tilde{\theta}} + \left(\frac{1}{\epsilon}-1 \right)q_v - q_l- q_i\right]
\end{equation}
in which, $q_v$ is the mixing ratio of water vapor, $q_l$ and $q_i$ are that of liquid cloud water and ice, respectively.

With this system of equations, at each time step, we can march equations (1) and (4) first. Note that from (1) we cannot separate $\tilde{\rho}$ and $\tilde{\theta}$, we need use the equation of state and hydrostatic balance equation. The reference state satisfies the equation of state such that 
\begin{equation}
\tcboxmath{
 \tilde{\pi} = \left( \frac{R}{p_s} \tilde{\rho} \tilde{\theta} \right)^{R/c_v}
 }
 \label{eqn:state}
\end{equation}
Then we can derive $\tilde{\theta}$ from the hydrostatic balance equation,
\begin{equation}
 \tcboxmath{
 c_p \tilde{\theta} \frac{\partial \tilde{\pi}}{\partial z} = -g 
 }
 \label{eqn:hystatic}
\end{equation}
and $\tilde{\rho}$ can be obtained after knowing $\tilde{\theta}$. \textit{The effect of moisture has been ignored equation (\ref{eqn:state}) and (\ref{eqn:hystatic}).}

The last variable we still do not know for the new time step is the pressure perturbation $\pi'$, which needs to be solved diagnostically to enforce Equation (1). The diagnostic relationship is obtained by multiplying the momentum equation by $\tilde{\rho}\tilde{\theta}$, taking the divergence of the result and subtracting $\partial/\partial t$ of Equation (1). The resulting diagnostic equation is provided by Durran (2008) as his Equation (5.2)
\begin{equation}
 \tcboxmath{
 \begin{aligned}
  c_p\nabla\cdot(\tilde{\rho}\tilde{\theta}\theta\nabla\pi') =& -\nabla\cdot(\tilde{\rho}\tilde{\theta}\mathbf{u}\cdot\nabla)\mathbf{u} - f\nabla_h \cdot (\mathbf{k}\times \tilde{\rho}\tilde{\theta}\mathbf{u}_h) + \frac{\partial\,\tilde{\rho}\tilde{\theta}B}{\partial z}\\
  &  - c_p\nabla_h\cdot(\tilde{\rho}\tilde{\theta}\theta \nabla_h\tilde{\pi}_h)
  - \frac{\partial }{\partial t}\left(\frac{\tilde{\rho}H_m}{c_p\tilde{\pi}} \right) +
  \nabla\cdot\left(\frac{\partial\,\tilde{\rho}\tilde{\theta}}{\partial t}\mathbf{u}\right) +
   \frac{\partial^2 \tilde{\rho}\tilde{\theta}}{\partial t^2}  \equiv \mathcal{R}
 \end{aligned}}
\end{equation}

The last term on the right-hand-side requires us to use a two-step time integration method. The Asselin leapfrog scheme seems to be a good candidate for this. At time level $(n-1)$ we are supposed to know every state varialbe, $(\tilde{\rho},\tilde{\theta},\tilde{\pi},\mathbf{u},\theta',\pi')$; At time level $n$ we know everything except $\pi'$, which does not own a prognostic equation. However, the thermodynamic variable Equations (1) and (4) can be integrated foward to yield varialbes at time level $n+1$. Without applying the Asselin filter, we can compute $\partial^2\tilde{\rho}\tilde{\theta}/\partial t^2$ from them, and thereby, using unfiltered variables, we obtain $\mathcal{R}$ at time level $n$. Solving the equation yields $\pi'$ at time level $n$ and allows us to advance the momentum equations. Note that in Durran (2008), there was no the term $\nabla\cdot(\partial\tilde{\rho}\tilde{\theta}/\partial t)\mathbf{u}$. It is unsure whether that is an error of paper or intentionally ignored. 


The algorithm can be summarised as follows, in which overline denote Asselin-filtered variable. The integration from $n=0$ to $n=1$ is ignored.

\begin{algorithm}
\begin{tcolorbox}[parbox=false, width=38em]

    Define model state $\mathbf{\Phi} = (\tilde{\rho},\tilde{\theta},\tilde{\pi},\theta,u,v,w) = (\mathbf{\Theta}, \mathbf{U})$\\[-1ex]         
    \qquad where $\mathbf{\Theta}$, $\mathbf{U}$ are thermodynamic and momentum state arrays.\\
    
    \For{ time level n = 1 ... N } {
        (i) update thermodynamics: $\mathbf{\Theta}_{n+1} = \overline{\mathbf{\Theta}}_{n-1} + 2\Delta t \mathcal{F}_{\Theta}(\mathbf{\Phi}_n)$,\\[-1ex]
        \qquad where $\mathcal{F}_{\Theta}$ is the tendency function for thermodynamic variables%,\\[-1ex]
        %\qquad\ and apply the Asselin filter 
        
        (ii) compute $\mathcal{R}(\overline{\mathbf{\Theta}}_{n-1}, \mathbf{\Theta}_{n}, \mathbf{\Theta}_{n+1}, \mathbf{U}_n)$ and solve the $\pi'_{n}$ equation\\[-1ex]
        \qquad where three time levels are needed for calculating $\partial^2\tilde{\rho}\tilde{\theta}/\partial t^2$. 
        
        (iii) advance momentum equations: $\mathbf{U}_{n+1} = \overline{\mathbf{U}}_{n-1} + 2\Delta t \mathcal{F}_{U}(\mathbf{\Phi}_n, \pi'_n)$,\\[-1ex]
        \qquad  where $\mathcal{F}_{U}$ is the tendency function for momentum variables,\\[-1ex]
        
        (iv) apply the Asselin filter\\[-1ex] 
\qquad $\overline{\mathbf{U}}_n = \mathbf{U}_n + \gamma (\overline{\mathbf{U}}_{n-1} - 2\mathbf{U}_n + \mathbf{U}_{n+1})$\\[-1ex]    
\qquad $\overline{\mathbf{\Theta}}_n = \mathbf{\Theta}_n + \gamma (\overline{\mathbf{\Theta}}_{n-1} - 2\mathbf{\Theta}_n + \mathbf{\Theta}_{n+1})$        

        (v) stack and continue: $\overline{\mathbf{\Phi}}_n = (\overline{\mathbf{\Theta}}_n, \overline{\mathbf{U}}_n)$; $\mathbf{\Phi}_{n+1} = (\mathbf{\Theta}_{n+1}, \mathbf{U}_{n+1})$
        }        
\end{tcolorbox}    
\end{algorithm}


The adjustment on $\pi'$ with a constant suggested by Durran (2008) or other means shall be applied every step. In real code, the iteration shall be done with a JAX \textsf{scan}.


\subsection*{References}
Durran, Dale. (2008). A physically motivated approach for filtering acoustic waves from the equations governing compressible stratified flow. \textit{Journal of Fluid Mechanics}, 601, 365-379. doi:10.1017/S0022112008000608.








\end{document}          
