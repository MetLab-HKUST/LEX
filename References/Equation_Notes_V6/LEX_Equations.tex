\documentclass[a4paper,11pt]{article}

\usepackage[left=2.5cm,top=2.5cm,right=2.5cm,bottom=2.5cm]{geometry}
\usepackage{graphicx}
\usepackage[font=footnotesize,labelfont=bf]{caption}
\usepackage{enumitem}
\usepackage{xcolor}
\usepackage{algorithm2e}

%\usepackage{amsmath,amsthm} 
%\usepackage{newtxtext}
%\usepackage{newtxmath}
% 
% \usepackage[p,osf]{scholax}
% T1 and textcomp are loaded by package. Change that here, if you want
% load sans and typewriter packages here, if needed
% \usepackage{amsmath,amsthm}% must be loaded before newtxmath
% amssymb should not be loaded
% \usepackage[scaled=1.075,ncf,vvarbb]{newtxmath}% need to scale up math package
% vvarbb selects the STIX version of blackboard bold.

% \usepackage[T1]{fontenc}
% \usepackage{stix}

\usepackage{kpfonts}
\usepackage[T1]{fontenc}

% 
% \usepackage{libertinus}
% \usepackage[T1]{fontenc}
% \renewcommand*\familydefault{\sfdefault} %% Only if the base font of the document is to be sans serif

\setlength{\parskip}{3ex}
\setlength{\parindent}{1.5em}
\renewcommand{\baselinestretch}{1.5}

\usepackage{amsmath,amsthm}
\usepackage[most]{tcolorbox}

\tcbset{colback=yellow!10!white, colframe=red!50!black, 
        highlight math style= {enhanced, %<-- needed for the ’remember’ options
            colframe=red,colback=red!10!white,boxsep=0pt}
        }
        

% Title Page
\title{\textbf{LEX Governing Equations}}
\author{Xiaoming Shi}

\begin{document}
\maketitle

The governing equations we adopted are the acoustic-wave-filtered equations for compressible stratified flow by Durran (2008). A pseudo-density $\rho^*$ is defined to eliminate sound waves. Mass conservation is enforced with respect to this pseudo-density such at
\begin{equation}
 \frac{1}{\rho^*} \frac{D \rho^*}{D t} + \nabla\cdot \mathbf{u} = 0 \nonumber
\end{equation}
The pseudo-density can be defined as 
\begin{equation}
 \rho^* = \frac{\tilde{\rho}(x,y,z,t)\tilde{\theta}_{\rho}(x,y,z,t)}{\theta_{\rho}} \nonumber 
\end{equation}
where $\tilde{\ }$ denotes a spatially varying reference state. Durran (2008) used potential temperature in their definition. Here, we replaced it with density potential temperature $\theta_{\rho}$ to include the effect of water variables. It is defined and approximated as
\begin{equation}
 \theta_{\rho} = \theta\left( \frac{1+ q_v / \varepsilon}{1+q_v+q_l+q_i}  \right)
 \approx \theta\left[ 1+\left(\frac{1}{\epsilon} - 1 \right)q_v - q_l - q_i\right] 
\end{equation}
where $q_v$, $q_l$, $q_i$ are mixing ratios of water vapor, liquid water, and cloud ice. In the reference state,
we can assume that $q_l = q_i = 0$, thus $\tilde{\theta}_{\rho}$ is the reference state virtual potential temperature.
With this definition, the mass (pseudo-density) conservation equation becomes, with some approximation,
\begin{equation}
%\tcboxmath{
\frac{\partial \tilde{\rho} \tilde{\theta}_{\rho}}{\partial t} + \nabla\cdot (\tilde{\rho}\tilde{\theta}_{\rho}\mathbf{u}) = \frac{\tilde{\rho} H_m}{c_p \tilde{\pi}}
%}
\end{equation}
in which, $H_m$ is the heating rate per unit mass. 

Define perturbations with respect to the reference state such that $\theta' = \theta - \tilde{\theta}$ and $\pi' = \pi - \tilde{\pi}$. Durran (2008) further separated $\tilde{\pi}$ into a large horizontally uniform component $\tilde{\pi}_v(z,t)$ and a remainder $\tilde{\pi}_h(x,y,z,t)$ for computational accuracy and notational convenience. Then the momentum and thermodynamics equations are the following,
\begin{equation}
\tcboxmath{
\frac{D \mathbf{u}_h}{D t}  + f {\mathbf{k}} \times \mathbf{u}_h +  c_p \theta_{\rho}\,\nabla_h(\tilde{\pi}_h + \pi') = 0
}
\end{equation}
\begin{equation}
 \tcboxmath{
 \frac{D w}{D t} + c_p \theta_{\rho} \frac{\partial \pi'}{\partial z} = B%g\frac{\theta'}{\tilde{\theta}}
 }
\end{equation}
\begin{equation}
\tcboxmath{
 \frac{D \theta }{D t} = \frac{H_m}{c_p \tilde{\pi}}
 }
\end{equation}
where $\mathbf{u}_h$ is the horizontal velocity vector, $\nabla_h$ is the horizontal gradient operator, and $f$ is the Coriolis parameter. $B$ is the linearized bouancy,
\begin{equation}
B = g\left[\frac{\theta'}{\tilde{\theta}} + \left(\frac{1}{\epsilon}-1 \right)(q_v-\tilde{q}_{v}) - q_l- q_i\right]
\end{equation}
in which, $\tilde{q}_v$ is the reference state mixing ratio of water vapor.
The reference state satisfies the equation of state and the hydrostatic balance equation,
\begin{equation}
\tcboxmath{
 \tilde{\pi} = \left( \frac{R}{p_s} \tilde{\rho} \tilde{\theta}_{\rho} \right)^{R/c_v}
 }
 \label{eqn:state}
\end{equation}
\begin{equation}
 \tcboxmath{
 c_p \tilde{\theta}_{\rho} \frac{\partial \tilde{\pi}}{\partial z} = -g 
 }
 \label{eqn:hystatic}
\end{equation}


The last variable we still do not know for integration is the pressure perturbation $\pi'$, which needs to be solved diagnostically to enforce Equation (2). The diagnostic relationship is obtained by multiplying the momentum equation by $\tilde{\rho}\tilde{\theta}_{\rho}$, taking the divergence of the result and subtracting $\partial/\partial t$ of Equation (2). The resulting diagnostic equation is provided by Durran (2008) as his Equation (5.2)
\begin{equation}
% \tcboxmath{
 \begin{aligned}
  c_p\nabla\cdot(\tilde{\rho}\tilde{\theta}_{\rho}\theta_{\rho}\nabla\pi') =& -\nabla\cdot(\tilde{\rho}\tilde{\theta}_{\rho}\mathbf{u}\cdot\nabla)\mathbf{u} - f\nabla_h \cdot (\mathbf{k}\times \tilde{\rho}\tilde{\theta}_{\rho}\mathbf{u}_h) + \frac{\partial\,\tilde{\rho}\tilde{\theta}_{\rho}B}{\partial z}\\
  &  - c_p\nabla_h\cdot(\tilde{\rho}\tilde{\theta}_{\rho}\theta_{\rho} \nabla_h\tilde{\pi}_h)
  - \frac{\partial }{\partial t}\left(\frac{\tilde{\rho}H_m}{c_p\tilde{\pi}} \right) +
  \nabla\cdot\left(\frac{\partial\,\tilde{\rho}\tilde{\theta}_{\rho}}{\partial t}\mathbf{u}\right) +
   \frac{\partial^2 \tilde{\rho}\tilde{\theta}_{\rho}}{\partial t^2}  % \equiv \mathcal{R}
 \end{aligned}%}
\end{equation}
Assuming the tendency in the reference state is small, we can ingore the last few terms involving time derivative in the equation aobve, then the diagnostic relation for $\pi'$ is
\begin{equation}
 \tcboxmath{
  c_p\nabla\cdot(\tilde{\rho}\tilde{\theta}_{\rho}\theta_{\rho}\nabla\pi') = -\nabla\cdot(\tilde{\rho}\tilde{\theta}_{\rho}\mathbf{u}\cdot\nabla)\mathbf{u} - f\nabla_h \cdot (\mathbf{k}\times \tilde{\rho}\tilde{\theta}_{\rho}\mathbf{u}_h) + \frac{\partial\,\tilde{\rho}\tilde{\theta}_{\rho}B}{\partial z}
   - c_p\nabla_h\cdot(\tilde{\rho}\tilde{\theta}_{\rho}\theta_{\rho} \nabla_h\tilde{\pi}_h) = \mathcal{R}}
\end{equation}


For time integration, we use the Asselin leapfrog scheme. At time level $(n-1)$ we are supposed to know every state varialbe, $(\tilde{\rho},\tilde{\theta},\tilde{\pi},\mathbf{u},\theta',\pi',q_v)$; At time level $n$ we know everything except $\pi'$, which does not own a prognostic equation. The thermodynamic variable Equations (5) can be integrated  foward to yield varialbes at time level $n+1$ first. Then, we compuate advection, Coriolis force, and reference state pressure gradient in the momentum equations to obtain the right-hand-side of (10) at time level $n$ and solve the equation to yield $\pi'$ at time level $n$. Finally we advance the momentum equations.


The algorithm can be summarised as follows, in which overline denote Asselin-filtered variable. The integration from $n=0$ to $n=1$ is integrated with a fourth-order Runge-Kutta method. 

\begin{algorithm}
\begin{tcolorbox}[parbox=false, width=38em]

    Define model state $\mathbf{\Phi} = (\tilde{\rho},\tilde{\theta},\tilde{\pi},\theta,q_v,u,v,w) = (\mathbf{\Theta}, \mathbf{U})$\\[-1ex]         
    \qquad where $\mathbf{\Theta}$, $\mathbf{U}$ are thermodynamic and momentum state arrays.\\
    
    \For{ time level n = 1 ... N } {
        (i) update thermodynamics: $\mathbf{\Theta}_{n+1} = \overline{\mathbf{\Theta}}_{n-1} + 2\Delta t \mathcal{F}_{\Theta}(\mathbf{\Phi}_n)$,\\[-1ex]
        \qquad where $\mathcal{F}_{\Theta}$ is the tendency function for thermodynamic variables%,\\[-1ex]
        %\qquad\ and apply the Asselin filter 
        
        (ii) compute $\mathcal{R}(\mathbf{\Theta}_{n}, \mathbf{U}_n)$ and solve the $\pi'_{n}$ equation.\\[-1ex]
        
        
        (iii) advance momentum equations: $\mathbf{U}_{n+1} = \overline{\mathbf{U}}_{n-1} + 2\Delta t \mathcal{F}_{U}(\mathbf{\Phi}_n, \pi'_n)$,\\[-1ex]
        \qquad  where $\mathcal{F}_{U}$ is the tendency function for momentum variables,\\[-1ex]
        
        (iv) apply the Asselin filter and continue to the next step\\[-1ex] 
\qquad $\overline{\mathbf{U}}_n = \mathbf{U}_n + \gamma (\overline{\mathbf{U}}_{n-1} - 2\mathbf{U}_n + \mathbf{U}_{n+1})$\\[-1ex]    
\qquad $\overline{\mathbf{\Theta}}_n = \mathbf{\Theta}_n + \gamma (\overline{\mathbf{\Theta}}_{n-1} - 2\mathbf{\Theta}_n + \mathbf{\Theta}_{n+1})$        
        }        
\end{tcolorbox}    
\end{algorithm}

There can be an adjustment on $\pi'$ with a constant. Durran (2008) has some suggestions but mass conservation can be an option. However, this correction is not needed if we do not need to know the exact density (no cloud process).

\subsection*{References}
Durran, Dale. (2008). A physically motivated approach for filtering acoustic waves from the equations governing compressible stratified flow. \textit{Journal of Fluid Mechanics}, 601, 365-379. doi:10.1017/S0022112008000608.








\end{document}          
