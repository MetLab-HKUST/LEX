\documentclass[a4paper,11pt]{article}

\usepackage[left=2.5cm,top=2.5cm,right=2.5cm,bottom=2.5cm]{geometry}
\usepackage{graphicx}
\usepackage[font=footnotesize,labelfont=bf]{caption}
\usepackage{enumitem}
\usepackage{xcolor}
\usepackage{algorithm2e}

%\usepackage{amsmath,amsthm} 
%\usepackage{newtxtext}
%\usepackage{newtxmath}
% 
% \usepackage[p,osf]{scholax}
% T1 and textcomp are loaded by package. Change that here, if you want
% load sans and typewriter packages here, if needed
% \usepackage{amsmath,amsthm}% must be loaded before newtxmath
% amssymb should not be loaded
% \usepackage[scaled=1.075,ncf,vvarbb]{newtxmath}% need to scale up math package
% vvarbb selects the STIX version of blackboard bold.

% \usepackage[T1]{fontenc}
% \usepackage{stix}

\usepackage{kpfonts}
\usepackage[T1]{fontenc}

% 
% \usepackage{libertinus}
% \usepackage[T1]{fontenc}
% \renewcommand*\familydefault{\sfdefault} %% Only if the base font of the document is to be sans serif

\setlength{\parskip}{3ex}
\setlength{\parindent}{1.5em}
\renewcommand{\baselinestretch}{1.5}

\usepackage{amsmath,amsthm}
\usepackage[most]{tcolorbox}

\tcbset{colback=yellow!10!white, colframe=red!50!black, 
        highlight math style= {enhanced, %<-- needed for the ’remember’ options
            colframe=red,colback=red!10!white,boxsep=0pt}
        }
        

% Title Page
\title{\textbf{LEX Dynamical Core}}
\author{Xiaoming Shi}

\begin{document}
\maketitle

The Large-Eddy simulation model in JAX (LEX) has two options for its governing equations. One is a set of pseudo-incompressible equations, which filter acoustic waves; the other is a set of fully compressible equations. The former is good for simulations in which the reference state of the flow does not change significantly during the simulation, and the latter is suitable for evolving flows.

\section{Acoustic-Wave-Filtered Equations}

In this option (\texttt{solver\_opt$\,$=$\,$1}), the governing equations we adopted are the acoustic-wave-filtered equations for compressible stratified flow by Durran (2008). A pseudo-density $\rho^*$ is defined to eliminate sound waves. Mass conservation is enforced with respect to this pseudo-density such at
\begin{equation}
 \frac{1}{\rho^*} \frac{D \rho^*}{D t} + \nabla\cdot \mathbf{u} = 0 \nonumber
\end{equation}
The pseudo-density can be defined as 
\begin{equation}
 \rho^* = \frac{\tilde{\rho}(x,y,z,t)\tilde{\theta}_{\rho}(x,y,z,t)}{\theta_{\rho}} \nonumber 
\end{equation}
where $\tilde{\ }$ denotes a spatially varying reference state. Durran (2008) used potential temperature in their definition. Here, we replaced it with density potential temperature $\theta_{\rho}$ to include the effect of water variables. It is defined and approximated as
\begin{equation}
 \theta_{\rho} = \theta\left( \frac{1+ q_v / \varepsilon}{1+q_v+q_l+q_i}  \right)
 \approx \theta\left[ 1+\left(\frac{1}{\epsilon} - 1 \right)q_v - q_l - q_i\right] 
\end{equation}
where $q_v$, $q_l$, $q_i$ are mixing ratios of water vapor, liquid water, and cloud ice. In the reference state,
we can assume that $q_l = q_i = 0$, thus $\tilde{\theta}_{\rho}$ is the reference state virtual potential temperature.
With this definition, the mass (pseudo-density) conservation equation becomes, with some approximation,
\begin{equation}
%\tcboxmath{
\frac{\partial \tilde{\rho} \tilde{\theta}_{\rho}}{\partial t} + \nabla\cdot (\tilde{\rho}\tilde{\theta}_{\rho}\mathbf{u}) = \frac{\tilde{\rho} H_m}{c_p \tilde{\pi}}
%}
\end{equation}
in which, $H_m$ is the heating rate per unit mass. 

Define perturbations with respect to the reference state such that $\theta' = \theta - \tilde{\theta}$ and $\pi' = \pi - \tilde{\pi}$. Then the momentum and thermodynamics equations are the following,
\begin{equation}
\tcboxmath{
\frac{\partial \mathbf{u}_{\text{h}}}{\partial t}  +  c_p \theta_{\rho}\,\nabla_{\text{h}}(\tilde{\pi} + \pi') =  - \mathbf{u}\cdot\nabla\mathbf{u}_{\text{h}} -f {\mathbf{k}} \times \mathbf{u}_{\text{h}} \equiv R_{\mathbf{u}_{\text{h}}}
}
\end{equation}
\begin{equation}
 \tcboxmath{
 \frac{\partial w}{\partial t} + c_p \theta_{\rho} \frac{\partial \pi'}{\partial z} = -\mathbf{u}\cdot\nabla w + B \equiv R_w
 }
\end{equation}
\begin{equation}
\tcboxmath{
 \frac{\partial \theta }{\partial t} = -\mathbf{u}\cdot \nabla \theta + \frac{H_m}{c_p \tilde{\pi}} \equiv R_{\theta}
 }
\end{equation}
where $\mathbf{u}_{\text{h}}$ is the horizontal velocity vector, $\nabla_{\text{h}}$ is the horizontal gradient operator, and $f$ is the Coriolis parameter. $B$ is buoyancy,
\begin{equation}
B = g\frac{\theta_{\rho} - \tilde{\theta}_{\rho}}{\tilde{\theta}_{\rho}} \approx g\left[\frac{\theta'}{\tilde{\theta}} + \left(\frac{1}{\epsilon}-1 \right)(q_v-\tilde{q}_{v}) - q_l- q_i\right]
\end{equation}
in which, $\tilde{q}_v$ is the reference state mixing ratio of water vapor.
The reference state satisfies the equation of state and the hydrostatic balance equation,
\begin{equation}
% \tcboxmath{
 \tilde{\pi} = \left( \frac{R}{p_s} \tilde{\rho} \tilde{\theta}_{\rho} \right)^{R/c_v}
% }
 \label{eqn:state}
\end{equation}
\begin{equation}
% \tcboxmath{
 c_p \tilde{\theta}_{\rho} \frac{\partial \tilde{\pi}}{\partial z} = -g 
% }
 \label{eqn:hystatic}
\end{equation}
If the reference state is horizontally uniform, which is usually the case, we can neglect $\tilde{\pi}$ in Equation (3).


The last variable we still do not know for integration is the pressure perturbation $\pi'$, which needs to be solved diagnostically to enforce Equation (2). The diagnostic relationship is obtained by multiplying the momentum equations by $\tilde{\rho}\tilde{\theta}_{\rho}$, taking the divergence of the result and subtracting $\partial/\partial t$ of Equation (2). The resulting diagnostic equation is provided by Durran (2008) as his Equation (5.2)
\begin{equation}
% \tcboxmath{
 \begin{aligned}
  c_p\nabla\cdot(\tilde{\rho}\tilde{\theta}_{\rho}\theta_{\rho}\nabla\pi') =& -\nabla\cdot(\tilde{\rho}\tilde{\theta}_{\rho}\mathbf{u}\cdot\nabla)\mathbf{u} - f\nabla_{\text{h}} \cdot (\mathbf{k}\times \tilde{\rho}\tilde{\theta}_{\rho}\mathbf{u}_{\text{h}}) + \frac{\partial\,\tilde{\rho}\tilde{\theta}_{\rho}B}{\partial z}\\
  &  - c_p\nabla_{\text{h}}\cdot(\tilde{\rho}\tilde{\theta}_{\rho}\theta_{\rho} \nabla_{\text{h}}\tilde{\pi})
  - \frac{\partial }{\partial t}\left(\frac{\tilde{\rho}H_m}{c_p\tilde{\pi}} \right) +
  \nabla\cdot\left(\frac{\partial\,\tilde{\rho}\tilde{\theta}_{\rho}}{\partial t}\mathbf{u}\right) +
   \frac{\partial^2 \tilde{\rho}\tilde{\theta}_{\rho}}{\partial t^2}  % \equiv \mathcal{R}
 \end{aligned}%}
\end{equation}
Assuming the tendency in the reference state is small, we can ignore the last few terms involving time derivative in the equation above, then the diagnostic relation for $\pi'$ is
\begin{equation}
 \tcboxmath{
 \begin{aligned}
  c_p\nabla\cdot(\tilde{\rho}\tilde{\theta}_{\rho}\theta_{\rho}\nabla\pi') = &-\nabla\cdot(\tilde{\rho}\tilde{\theta}_{\rho}\mathbf{u}\cdot\nabla)\mathbf{u} - f\nabla_{\text{h}} \cdot (\mathbf{k}\times \tilde{\rho}\tilde{\theta}_{\rho}\mathbf{u}_{\text{h}}) + \frac{\partial\,\tilde{\rho}\tilde{\theta}_{\rho}B}{\partial z}
   \\ &- c_p\nabla_{\text{h}}\cdot(\tilde{\rho}\tilde{\theta}_{\rho}\theta_{\rho} \nabla_{\text{h}}\tilde{\pi})
  \end{aligned}
   }
\end{equation}
where the last term due to the horizontal gradient of reference state Exner pressure can be neglected if we have a horizontally uniform reference state. 

This equation can be solved with the Bi-Conjugate Gradient Stable iteration (BiCGSTAB) (\texttt{jax.scipy.sparse.linalg.bicgstab}) after discretization. There can be an adjustment on $\pi'$ with a constant. Durran (2008) has some suggestions but mass conservation can be an option. However, the perturbation Exner pressure is usually very small, so when we need density (such as for cloud process), we can use the reference state density as a good approximation.


\textbf{Equations (3), (4), (5), and (10) comprise the governing equations for the acoustic-wave-filtered dynamical core.} We integrate them with the third-order Runge-Kutta (RK3) described by Wicker and Skamarock (2002). This scheme takes three steps to advance a solution $\Phi(t)$ to $\Phi(t+\Delta t)$:
\begin{subequations}
\begin{align}
\Phi^* &= \Phi^t + \frac{\Delta t}{3} F (\Phi^t) \\
\Phi^{**} &= \Phi^t + \frac{\Delta t}{2} F (\Phi^*) \\
\Phi^{t+\Delta t} &= \Phi^t + \Delta t F(\Phi^{**})
\end{align}
\end{subequations}
where $\Delta t$ is the time step for one full step of the RK3 integration. During the integration, we compute terms need for solving the pressure equation first, such as advection and buoyancy, and then solve Equation (10). Note that we need to use $\Phi^t$, $\Phi^*$, $\Phi^{**}$ for three RK3 substeps, respectively, to compute the right-hand-side Equation (10). After having the Exner pressure, we can solve the momentum equations according to (11). The potential temperature and water variable equations are integrated last. 


\section{Fully Compressible Equations}

When the reference state should evolve with time, the pseudo-incompressible equations can pose a significant limit to a simulation. Therefore, LEX also supports solving the fully compressible equations with 
\texttt{solver\_opt$\,$=$\,$2} and \texttt{solver\_opt}$\,$=$\,$3. We have the same momentum and potential temperature equations as in the pseudo-incompressible solver. However, we have a prognostic equation for the Exner pressure:
\begin{equation}
 \tcboxmath{
 \frac{\partial \pi'}{\partial t} + \mathbf{u}\cdot\nabla\pi + \frac{R}{c_v}\pi\nabla\cdot\mathbf{u} = R_{\pi}
 }
\end{equation}
The term $R_{\pi}$ represents the effects of diabatic heating and phase changes of water vapor. It is not uncommon to neglect it. In the current LEX code, it is just a placeholder variable and set to 0.


The fully compressible equations allows acoustic waves and we use the time splitting method of Wicker \& Skamarock (2008) to use small acoustic steps during each RK3 substep to integrate terms related to sound waves. The slow mode terms are evaluated according to $\Phi^t$, $\Phi^*$, or, $\Phi^{**}$ at the beginning of each RK3 substep, and then a number of small acoustic steps are use to integrate the velocities and Exner pressure from $t$ to $t+\Delta t/3$, $t+\Delta t/2$, or, $t + \Delta t$ in each RK3 substep by combining the slow-mode terms and fast-mode terms, the latter of which are evaluated at the acoustic step time $\tau$. The integration methods used in LEX is the same as CM1.


\subsection{Explicit Time-Splitting Solver}

The explicit time-splitting method (\texttt{solver\_opt$\,$=$\,$2}) described by Wicker \& Skamarock (2002) integrates the momentum and Exner pressure equation explicitly in both horizontal and vertical. The temporally discretized equations used in the acoustic steps are the following:
\begin{align}
\frac{\mathbf{u}_{\text{h}}^{\tau+\Delta \tau} - \mathbf{u}_{\text{h}}^{\tau}}{\Delta\tau} &=   
    -c_p\theta_{\rho}^{t*} \nabla_{\text{h}}(\tilde{\pi} + \pi'^{\tau}) + R_{\mathbf{u}_{\text{h}}}^{t*}\\
\frac{w^{\tau+\Delta\tau} - w^{\tau}}{\Delta\tau} &= 
-c_p\theta_{\rho}^{t*} \frac{\partial\pi'^{\tau}}{\partial z} + R_w^{t*}\\
\frac{\pi'^{\tau+\Delta\tau} - \pi'^{\tau}}{\Delta\tau} &=-\mathbf{u}^{\tau+\Delta\tau}\cdot\nabla\pi^{t*} -\frac{R}{c_v}\pi^{t*}\nabla\cdot\mathbf{u}^{\tau+\Delta\tau} + R_{\pi}^{t*}
\end{align}
In those equations, superscript $\tau$ denote the present time of the acoustic step and $\Delta \tau$ is small acoustic time step. The superscript $t*$ denotes variables used for evaluate the current RK3 substep tendency, which are used to evaluate slow-mode terms. For the first (11a), second (11b), and third (11c) RK3 substeps, the superscript $t*$ should be replaced by $t$, $*$, and $**$, respectively. 

In each RK3 substep, the acoustic mode integration starts from time $t$ and integrates to 
$t+\Delta t/3$, $t+\Delta t/2$, $t+\Delta t$ for the three RK3 substeps, respectively. Therefore, if we denote the number of acoustic steps in one full RK3 step ($\Delta t$) as $n_s$, the numbers of acoustic steps for the three RK3 substeps are $n_s/3$, $n_s/2$, and $n_s$, respectively. However, Wicker \& Skamarock (2002) found that for the first RK3 substep, it does not cause a significant loss of accuracy and stability to take a single small time step of $\Delta \tau=\Delta t/3$ in the first RK3 substep. Therefore, in the LEX code, the number of acoustic steps in the first RK3 substep is set to one. 

\subsection{Implicit Time-Splitting Solver}

When the vertical grid spacing is very small, the explicit time-splitting method can be very time consuming since the acoustic time step $\Delta \tau$ is limited by the vertical grid spacing. To alleviate this issue, LEX also implements an implicit time-splitting method (\texttt{solver\_opt$\,$=$\,$3}) described by Klemp and Wilhelmson (1978) for the vertical terms in the momentum and Exner pressure equations. The horizontal terms are still treated explicitly (Equation 13) as in the explicit time-splitting method. The temporally discretized $w$ and $\pi'$ equations used in the acoustic steps are the following:
\begin{align}
\frac{w^{\tau+\Delta\tau} - w^{\tau}}{\Delta\tau} =& 
-c_p\theta_{\rho}^{t*} \frac{\partial\overline{\pi'}^{\tau}}{\partial z} + R_w^{t*}\\
\frac{\pi'^{\tau+\Delta\tau} - \pi'^{\tau}}{\Delta\tau} =&
-\mathbf{u}_{\text{h}}^{\tau+\Delta\tau}\cdot\nabla_{\text{h}}\pi^{t*} -\overline{w}^{\tau}\frac{\partial\pi^{t*}}{\partial z} \nonumber \\ 
& -\frac{R}{c_v}\pi^{t*}(\nabla_{\text{h}}\cdot\mathbf{u}_{\text{h}}^{\tau+\Delta\tau} 
  +\frac{\partial \overline{w}^{\tau}}{\partial z}) 
 + R_{\pi}^{t*}
\end{align}
where the overline denotes the average between time levels $\tau$ and $\tau+\Delta \tau$, i.e.,
\begin{align}
 \overline{\pi'}^{\tau}& = \alpha\pi'^{\tau+\Delta\tau} + (1-\alpha)\pi'^{\tau} \\
 \overline{w}^{\tau} &= \alpha w^{\tau+\Delta\tau} + (1-\alpha)w^{\tau}
\end{align}
Choosing an $\alpha>0.5$ (forward-in-time weighting) provides numerical damping to vertically propagating sound waves, stabilizing the integration. In LEX, we set $\alpha=0.6$ as default.

To solve Equations (16) and (17), we substitute Equation (17) into (16) to eliminate $\overline{\pi'}^{\tau}$, resulting in a tridiagonal system for $w^{\tau+\Delta \tau}$:
\begin{equation}
A_k w^{\tau+\Delta \tau}_{k-1} + B_k w^{\tau+\Delta \tau}_k + C_k w^{\tau+\Delta \tau}_{k+1} = D_k
\end{equation}
where the subscript $k$ denotes the vertical level index.
Such a system can be obtained by using centered finite differences on the staggered C-grid in the vertical direction. Here, $D_k$ contains only known quantities at time level $\tau$, and the coefficients $A_k$, $B_k$, and $C_k$ contain known variables at time level $t*$. Those coefficients are:
\begin{align}
 A_k &= -\frac{\alpha^2 \Delta \tau^2 c_p \theta_{\rho}^{t*}}{\Delta z^w_k}\left(
\frac{R}{c_v}\frac{\pi^{t*}_{k-1}}{\Delta z_{k-1}} -
\frac{\pi_{k-1}^{t*} - \pi_{k-2}^{t*}}{2\Delta z_{k-1}^w}
\right) \\
B_k &= 1 + \frac{\alpha^2 \Delta \tau^2 c_p \theta_{\rho}^{t*}}{\Delta z_k^w} \frac{R}{c_v}
\left( 
\frac{\pi^{t*}_{k}}{\Delta z_{k}} + \frac{\pi_{k-1}^{t*}}{\Delta z_{k-1}}   
\right)\\
 C_k &= -\frac{\alpha^2 \Delta \tau^2 c_p \theta_{\rho}^{t*}}{\Delta z^w_k}\left(
\frac{R}{c_v}\frac{\pi^{t*}_{k}}{\Delta z_{k}} +
\frac{\pi_{k+1}^{t*} - \pi_{k}^{t*}}{2\Delta z_{k+1}^w}
\right)
\end{align}
Note that we have staggered C-grid, the subscript $k$ for $\pi$ indicate cell centers, while that for $w$ indicate the cell interface level. $\Delta z$ indicates the grid spacing centered at the cell centers, and $\Delta z^w$ is the centered at the $w$ levels. However, if we do not use vertically stretched grid, they are the same.  

The right-hand-side term $D_k$ is
\begin{equation}
 \begin{aligned}
  D_k &= w_k^{\tau} + \Delta\tau\left(R_w^{t*} - c_p\theta_{\rho}^{t*}\frac{\partial \pi'^{\tau}}{\partial z}\right)
  - \alpha \Delta\tau^2 c_p \theta_{\rho}^{t*}\frac{\partial}{\partial z} \left(R_{\pi}^{t*} 
  -\mathbf{u}^{\text{mix}}\cdot \nabla\pi^{t*}
  -\frac{R}{c_v}\pi^{t*}\nabla\cdot \mathbf{u}^{\text{mix}} 
  \right)
%    D_k &= w_k^{\tau} + \Delta\tau\left\{R_w^{t*} - c_p\theta_{\rho}^{t*}\frac{\partial}{\partial z}\left[
%  \pi'^{\tau}+ \alpha \Delta\tau \left( R_{\pi}^{t*} 
%  -\mathbf{u}^{\text{mix}}\cdot \nabla\pi^{t*}
%  -\frac{R}{c_v}\pi^{t*}\nabla\cdot \mathbf{u}^{\text{mix}} 
%  \right)\right]\right\}
 \end{aligned}
\end{equation}
where $\mathbf{u}^{\text{mix}} \equiv [u^{\tau+\Delta\tau},\ v^{\tau+\Delta\tau},\ (1-\alpha) w^{\tau}]$ is not a real velocity vector, but this notation makes the equation concise and reminds us that those terms involving it can be computed with existing advection and divergence functions. 

We solve the the tridiagonal system (20) from $k=2$ to $k=n_z$, where $k=1$ and $k=n_z+1$ are the bottom and ceiling $w$ levels at which $w=0$. After obtaining $w^{\tau+\Delta \tau}$, we can compute $\pi'^{\tau+\Delta \tau}$ from Equation (17). The rest of the integration procedure is the same as in the explicit time-splitting method.



\subsection*{References}

\begin{description}[topsep=0pt, partopsep=0ex, itemsep=0ex, parsep=0ex, leftmargin=*, labelindent=-0.19cm, itemindent=-25pt]

\item Durran, Dale. (2008). A physically motivated approach for filtering acoustic waves from the equations governing compressible stratified flow. \textit{Journal of Fluid Mechanics}, 601, 365-379. 

\item Klemp, J. B., \& Wilhelmson, R. B. (1978). The Simulation of Three-Dimensional Convective Storm Dynamics. \textit{Journal of Atmospheric Sciences}, 35(6), 1070-1096

\item Wicker, L. J., \& Skamarock, W. C. (2002). Time-splitting methods for elastic models using forward time schemes. \textit{Monthly Weather Review}, 130(8), 2088-2097.

\end{description}




\end{document}          
